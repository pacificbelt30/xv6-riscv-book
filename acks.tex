\chapter*{序文と謝辞}

この文書は,オペレーティングシステムの講義用テキストのドラフトで
す.xv6 というカーネルを例題にして,オペレーティングシステムの主なコン
セプトを説明します.xv6 は,Dennis Ritchie と Ken Thompson による Unix
Version 6 (v6)~\cite{unix} の再実装です.xv6 は,v6 の構造とスタイルを
ゆるく踏襲しますが,ANSI C~\cite{kernighan} で記述され,マルチコア
の RISC-V~\cite{riscv} で動作します.

このテキストは,xv6 のソースコードと一緒に読むことを想定しています.
``John Lions's Commentary on UNIX 6th Edition''~\cite{lions} 
に触発されたアプローチです.xv6 手を動かす実習課題を含む
v6 と xv6 オンライン教材へのリンクは,
\url{https://pdos.csail.mit.edu/6.828}
を見てください.

私たちはこのテキストを MIT のオペレーティングシステムの講義である6.828 で
利用しました.xv6 に直接的・間接的に貢献してくれた教員,ティーチングア
シスタント,そして 6.828 の受講生に感謝します.Austin
Clements と Nickolai Zeldovich には特に感謝をします.最後に,
テキストのバグや改善のための助言をメールを送ってくれた以下の人たちに感謝します: 
%
Abutalib Aghayev, Sebastian Boehm, Anton Burtsev, Raphael Carvalho,
Tej Chajed, Rasit Eskicioglu, Color Fuzzy, Giuseppe, Tao Guo, Robert
Hilderman, Wolfgang Keller, Austin Liew, Pavan Maddamsetti, Jacek
Masiulaniec, Michael McConville, miguelgvieira, Mark Morrissey, Harry
Pan, Askar Safin, Salman Shah, Ruslan Savchenko, Pawel Szczurko,
Warren Toomey, tyfkda, tzerbib, Xi Wang, and Zou Chang Wei.

間違いを見つけたり,改善のための提案があれば,
Frans Kaashoek と Robert Morris (kaashoek,rtm@csail.mit.edu)
にメールしてください.

